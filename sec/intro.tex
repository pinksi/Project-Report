\section{INTRODUCTION}
\pagenumbering{arabic} % Start arabic page numbers from here


\subsection{Background}
A stock market is a network of economic transaction of stocks that represents a claim of ownership on businesses and companies. To own the stock of a certain company simply means to own a part of the company and claim a right to participate in its earnings. A shareholder is someone who invests his/her money in a corporation expecting certain profit in return. It is therefore very important to evaluate a company's performance before buying its stocks. 
The ownership of a shareholder towards a company depends upon the volume of share s/he holds, with respect to the volume of outstanding share. The stock market runs on the concept of supply and demand. The stock of a company with a better profile, well known products and high profits is more in demand than a company with an average profile, lesser known products and low profits and therefore, costs more. However, unlike the trade of foreign currency, the prices of companies in the stock market fluctuate heavily, in a minute-to-minute basis. That is because, every buying or selling activity results in a heavy distortion in the supply and demand cycle. Thus, a person earning profit in a stock transaction ultimately means a loss in someone else's transaction. 

There are two types of market for stock transaction, \textbf{Primary Market and Secondary Market}. The primary market deals with new securities. It is the market for new long term equity capital, issued by the company directly to the investors for setting up new business or for expanding the existing business. \cite{wiki} In a primary market, companies, governments or public sector institutions can raise funds through bond issues and corporations can raise capital through the sale of new stock through an initial public offering (IPO). It performs a crucial function of facilitating capital formation in the economy and therefore, is also referred to as New Issue Market (NIM).The secondary market is the market where previously issued financial stocks are bought and sold. The secondary market is required to be highly liquid so that, both the buyer and the seller can get the actual worth of return from the transaction. For this purpose, marketplace centralization is a must. Exchanges such as Nepal Stock Exchange (NEPSE), New York Stock Exchange (NYSE) provide a centralized and liquid platform to serve as a secondary marketplace for the investors. 

\cite{sto} NEPSE is the one and only secondary marketplace in Nepal. Established before 1993 under the name 'Securities Exchange Center', NEPSE first opened its trading floor on 13th January, 1994. Since then, it has been working to fulfill its objective of imparting free marketability and liquidity to the government and corporate securities by facilitating transactions in its trading floor through stock brokers. As of now, more than 200 companies from Banking, Finance, Hydropower, Hotels, Development Banks Manufacturing and Co., Insurance and Trading sectors have registered in NEPSE. Stock market plays a vital role in a nation's economy. \cite{eco} A collapse in share prices has the potential to cause widespread economic disruption. The famous stock market crash of 1929 was a key factor in causing the great depression of the 1930s. Along with individual wealth, the stock market also has a huge impact on the well-being of financial corporations and firms. The ability of a company to perform well in the industrial sector is related directly with its ability to perform well in the stock market. If a company is going through a bearish trend, it may have to struggle financially. Moreover, the investors' trust in the company is also hampered in such cases. In turn, the overall performance of all industrial and financial firms associated with the stock market can have a huge impact in the overall economy of a country. Multi-national companies with worldwide investors have an even bigger impact in the economy of individuals in multiple countries. In the same way, the stock market trends also affect the investments in other sectors like: forex, gold and land investments.  

Hence, considering all the factors which affect and which are affected by the stock market, it can be said that the stock market prediction is a very sensitive topic. There are many theories such as Efficient Market Hypothesis (EMH) which claim that, outsmarting the market is impossible in all cases. Still, considerable efforts are being made in this sector using various artificial intelligence and deep learning techniques. 

\subsection{Objectives}
The main objectives of this projects are as follows :
\begin{itemize}
	\item To predict the stock market trend based on technical, fundamental and news-sentimental analysis
 	\item To visualize the prediction results and daily trading prices in the form of interactive charts
	\item To compare the results and effectiveness of the algorithms: artificial neural network with backpropagation, knn and naive bayes.
\end{itemize}

\subsection{Problem Statement}
Theoretically, the stock market is said to be very difficult to predict, due to its dynamic and non-linear model. However, the investors and stock analysts have been trying to somehow predict the stock prices of a company, to increase the profit in buying and selling stocks. Appreciable efforts have also been made from academic researchers and enthusiasts in this field. However, identifying the pattern of such an uncertain system through simple calculations and mathematics results in poor accuracy with questionable reliability. The overall hit rates of these methodologies and models are generally too low to be practical for real-world application. Complex, dedicated systems and models are required which can take into consideration, the numerous factors that can affect the stock price of a company. For an instance, the intrinsic valuation of a company and its performance in the market till now are equally important factors in determining its future price. However, it is very difficult to know for certain which factor affects the most at the given time, and by how much. Therefore, the market should be analyzed under various influencing factors, the prime of which are: Technical factors, Fundamental factors and News-sentimental factors. In technical analysis, the prediction model is built considering a company's past performance in the stock market, which includes studying the past rise and fall trends, average traded volumes, bullish and bearish trend behaviors and so on. It is based on the assumption that history repeats itself and that future market directions can be determined by examining the way the market has behaved before. Thus, it is assumed that price trends and patterns exist that can be identified and utilized for predictions. In fundamental analysis, the worth of the company, its current profits, capital gains and the future profits plays a vital role in understanding its stock price behaviors. In news-sentiment analysis, the immediate effects of political, economic and stock related news in a company's stock prices is studied and applied.

Therefore, through circumstantial application of above mentioned analysis, this project presents a general and complete solution for stock prediction, which can be employed in the real world for gaining profit in the stock market.


\subsection{Scope}
The project aims to predict the stock trend movements of trading companies based on large volume of historical data collected from various sources. The historical data constitutes of a company's fundamental valuations, past trading prices and volumes, and past news features. The basic driving factors for choosing a prediction model is its effectiveness, applicability and accuracy of results. By using multiple analysis and prediction models, the project aims to compare the usability of each such models. On completion of this project, we aim to establish a highly reliable stock prediction system, which can be used by investors to decide when to buy or sell the stocks of a company in order to gain maximum profit. It is hoped that the project will be beneficial for the stakeholders including, researchers, business analysts, stock market enthusiasts and policy makers. The project is also focused on improving the trading experience of new investors who may or may not know much about the market behaviors.
