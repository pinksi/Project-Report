\section{LITERATURE REVIEW}
\subsection{Theory Details}
Stock Market Prediction is a hot topic in data mining. Many analysts and researchers have done a lot of work in this field applying various data mining and statistical techniques to develop stock prediction models. The most popular methods used for stock market prediction are Dynamic time series model, Hidden Markov Model, Bayesian Classifiers, Artificial Neural Network, technical analysis, fundamental analysis and so on. All these methods work in different manner and work on different precision levels.
 
When predicting the future prices of Stock Market securities, there are several theories available. The first is Efficient Market Hypothesis (EMH). \cite{emh} In EMH, it is assumed that the price of a security reflects all of the information available and that everyone has some degree of access to the information. Fama's theory further breaks EMH into three forms: Weak, Semi-strong, and Strong. In Weak EMH, only historical information is embedded in the current price. The Semi-Strong form goes a step further by incorporating all historical and currently public information in the price. The Strong form includes historical, public, and private information, such as insider information, in the share price. From the tenets of EMH, it is believed that the market reacts instantaneously to any given news and that it is impossible to consistently outperform the market.

A different perspective on prediction comes from Random Walk Theory. \cite{mal} In this theory, Stock Market prediction is believed to be impossible where prices are determined randomly and outperforming the market is infeasible. Random Walk Theory has similar theoretical underpinnings to Semi-Strong EMH where all public information is assumed to be available to everyone. However, Random Walk Theory declares that even with such information, future prediction is ineffective.

It is from these theories that two distinct trading philosophies emerged; the fundamentalists and the technicians. In a fundamentalist trading philosophy, the price of a security can be determined through the nuts and bolts of financial numbers. These numbers are derived from the overall economy, the particular industry's sector, or most typically, from the company itself. Figures such as inflation, joblessness, return on equity (ROE), debt levels, and individual Price to Earnings (PE) ratios can all play a part in determining the price of a stock.

In contrast, technical analysis depends on historical and time-series data. These strategists believe that market timing is critical and opportunities can be found through the careful averaging of historical price and volume movements and comparing them against current prices. Technicians also believe that there are certain high/low psychological price barriers such as support and resistance levels where opportunities may exist. They further reason that price movements are not totally random, however, technical analysis is considered to be more of an art form rather than a science and is subject to interpretation.

Both fundamentalists and technicians have developed certain techniques to predict prices from financial news articles. In one model that tested trading philosophies, \cite{tra} LeBaron et. al. posited much can be learned from a simulated stock market with simulated traders. In their work, simulated traders mimicked human trading activity. Because of their artificial nature, the decisions made by these simulated traders can be dissected to identify key nuggets of information that would otherwise be difficult to obtain. The simulated traders were programmed to follow a rule hierarchy when responding to changes in the market; in this case it was the introduction of relevant news articles and/or numeric data updates. Each simulated trader was then varied on the timing between the point of receiving the information and reacting to it. The results were startling and found that the length of reaction time dictated a preference of trading philosophy. Simulated traders that acted quickly formed technical strategies, while traders that possessed a longer waiting period formed fundamental strategies. It is believed that the technicians capitalized on the time lag by acting on information before the rest of the traders, which lent this research to support a weak ability to forecast the market for a brief period of time

In similar research on real stock data and financial news articles, \cite{gido} Gidofalvi gathered over 5,000 financial news articles concerning 12 stocks, and identified this brief duration of time to be a period of twenty minutes before and twenty minutes after a financial news article was released. Within this period of time, Gidofalvi demonstrated that there exists a weak ability to predict the direction of a security before the market corrects itself to equilibrium. One reason for the weak ability to forecast is because financial news articles are typically reprinted throughout the various news wire services. Gidofalvi posits that a stronger predictive ability may exist in isolating the first release of an article. Using this twenty-minute window of opportunity and an automated textual news parsing system, the possibility exists to capitalize on stock price movements before human traders can act. 

Also Ralph Nelson Elliott developed the \cite{wav} Elliott wave theory in the late 1920s by discovering that stock markets, thought to behave in a somewhat chaotic manner, in fact traded in repetitive cycles. Elliott discovered that these market cycles resulted from investors' reactions to outside influences, or predominant of psychology of the masses at the time. He found that the upward and downward swings of the mass psychology always showed up in the same repetitive patterns, which were then divided further into patterns he termed "waves".

For the stock market prediction, Artificial Neural Network(ANN) has been considered the most efficient method. Inspired by neurosciences, ANNs have shown great potential in terms of recognizing patterns in nonlinear systems. Existing research suggests that ANN is an eminent model to predicting stock markets due to its dynamical characteristics. Even so, a common criticism of neural networks is that they require a large diversity of training for real-world operation. Moving average analysis and single exponential smoothing methods are frequently used in order to make stock analysis. The Nepal stock exchange (NEPSE) uses exponential smoothing in its website for this purpose. Moving averages work quite well in strong trending conditions, but often poorly in choppy or ranging conditions.

Under the assumption that the stock market could be predicted, there are some major cateogories of prediction methods: fundamental analysis, technical analysis and news analysis.


\subsection{Related Work}
Many algorithms of data mining have been proposed to predict stock price. Neural Network, Genetic Algorithm, Decision Tree and Fuzzy systems are widely used. In addition, pattern discovery is beneficial for stock market prediction and public sentiment is also related to predicting stock price. Projects have been done on predicting stock value based on Fuzzy logic.

There are a lot of software and web applications working with the similar concept. Nepal Sharemarket is a website that makes individual, comparative as well as in depth analysis on stock market companies and also forecasts their price on a chosen time basis. Another website, \textit{stockforecasting.com} also makes stock prediction using neural networks and boasts of highest accuracy among all the stock-prediction applications. It is an American company and gives minute predictions of various international companies.

\cite{iknow} \textit{Iknowfirst.com} uses predictive forecast algorithm based on artificial intelligence and machine learning with elements of Artificial Neural Networks and Genetic Algorithms incorporated in it. Its system outputs the predicted trend as a number, positive or negative, along with the wave chart that predicts how the waves will overlap the trend. This helps the trader decide which direction to trade, at what point to enter the trade, and when to exit.

\cite{itt} MetaStock is a widely acclaimed software, the edge it has is excellent news service, expert advisors and system development, with a huge range of indicators and powerful scanning. It can provide excellent earlier shorter term signals of trend changes that allow the investor to fine tune the trade. 

%In \cite{rs-review}, the authors have provided different characteristics and
